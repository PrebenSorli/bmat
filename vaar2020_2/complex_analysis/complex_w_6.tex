\title{Forelesninger Week 6 Complex Analysis}
\maketitle
\section{11. feb.}
$$f: \R \to \R, \quad x \mapsto
\begin{cases}
  0 \text{ if } x= 0 \\
  exp(-\frac{1}{x^2}) \text{ if }x \neq 0
\end{cases}
$$
$f$ is of class $C^{\infty}$ and $f^n(0)=0 \forall n \in \Z_{n\geq 0}$.
\newline $\sum \frac{f^{(n)}}{n!}x^n \equiv 0 \rightarrow $ $f$ is not locally given by a power series.

\subsection*{Taylor expansion of holomorphic functions}
Let $D\subseteq \C$ open, $f: D \to \C$ holomorphic, $r\in \R_{>0}, z_0 \in D$, and assume $D_r(z_0)\subseteq D$. Then
  \begin{enumerate}[(a)]
    \item On $D_r(z_0)$ the function $f$ is given as a power series centered at $z_0$, whose radius of convergence is $\geq r$. In particular, $f$ is infinitely many times complex differentiable and
      $$f(z)=\sum_{n=0}^{\infty}\frac{f^{(n)}}{n!}(z-z_0)^n, \forall z\in D_r(z_0)$$
    \item If $0<s<r$, then $\forall n \in \Z_{n\geq0 }$,
      $$f^{(n)}(z_0)= \frac{n!}{2\pi i}\int_{|w-z_0|=r}\frac{f(w)}{(w-z_0)^{n+1}}dw$$
  \end{enumerate}
  \begin{proof}
    Let $z\in D_r(z_0)$. Pick an $s\in (0,r)$ with $z-z-0|<s$ i.e. $z\in D_s(z_0)$. Since $D_s(z_0)\subseteq D_r$ we can apply the Cauchy Integral formula:
      $$f(z) = \frac{1}{2 \pi i} \int_{|w-z_0|=s}dw=\frac{1}{2 \pi i}\int_{|w-z_0|=s}\frac{f(w)}{w-z_0} \cdot \frac{1}{1- \frac{z-z_0}{w-z_0}}dw$$
      $$ \left|\frac{z-z_0}{w-z_0}\right| \leq 1 ??? $$
      This is a geometric series so
      $$= \frac{1}{2 \pi i} \int_{|w-z_0|=s}\frac{f(w)}{w-z_0}\sum_{n=0}^{\infty}\left(\frac{z-z_0}{w-z_0}   \right)^n dw$$
      $$(*)= \frac{1}{2\pi i} \sum_{n=0}^{\infty} \int_{|w-z_0|=s}\frac{f(w)}{w-z_0} \left(\frac{z-z_0}{w-z_0}\right)^n dw $$
      $$ = \sum_{n=0}^{\infty} \left[  \frac{1}{2 \pi i} \int_{|w-z_0|=s}\frac{f(w)}{(w-z_0)^{n+1}}dw \right] (z-z_0)^n $$
      The term inside the brackets is constant in $\C$, independent from $z$, and from $s\in(0, r)$ by homotopy invariance.
      This is a power series around $z_0$, convergence for $x\in D_r(z_0)$ implicit from calculation. By uniqueness of power series it follows that
        $$f^{(n)}(z_0)= \frac{n!}{2\pi i}\int_{|w-z_0|=s}\frac{f(w)}{(w-z_0)^{m+1}}dw$$
      Now we have to verify $(*)$: this follows straight away from the dominated convergence theorem:
        $$|f_n|= \left| \frac{f(re^{i\varphi}+z_0)}{(re^{i\varphi}+z_0)-z_0} \cdot \left(\frac{z-z_0}{re^{i\varphi}+z_0-z_0}\right)^n \cdot ire^{i\varphi}\right|=\left|\frac{f(re^{i\varphi}+z_0) \cdot(z-z_0)^n \cdot r }{re^{i\varphi} \cdot r^n(e^{i\varphi})^n}\right|$$
        $$= \left|\frac{z-z_0}{r}\right|^n \left|f(re^{i\varphi}+z_0)\right| \leq \left| f(re^{i\varphi}+z_0)\right| \leq max_{0 \leq \varphi \leq 2\pi}\left|f(re^{i\varphi}+z_0)\right|<\infty$$
          \qedhere
  \end{proof}
Assuming $f$ has a power sereies $f(z)=\sum_{n=0}^{\infty}a_n(z-z_0)^n$, we set by termwise differentiation that $a_n=\frac{f^{(n)}(z_0)}{n!}$

\begin{example}
  $$f: \C \setminus\{-i,i\} \to \C, Z \mapsto \frac{1}{z^2+1}$$
  $f$ is given as a power series centered in $D_{\sqrt{2}}(1)$.
\end{example}

\begin{remark}
    $D \subseteq \C$, $f: D\to \C$ holomorphic, $D_r(z_0)\subseteq D$ for some $z_0 \in D$, $r\in \{R_{>0}$. Then $\forall z \in D_r(z_0)$:
      $$\forall n \in \Z_{\geq 0}: f^{(n)}(z)=\frac{n!}{2\pi i} \int_{|w-z_0|=r}\frac{f(w)}{(w-z)^{n+1}}dw$$
    ``Generalized Cauchy Integral formula'' -- observe that $z$ now don't have to be the center of the disk $z_0$, but can take any value in the interior $D_r(z_0)$.
\end{remark}
\begin{proof}
  (Sketch)
  \newline The proof is by homotopy invariance. $s>0: D_s(z_0)\subseteq D_r(z_0)$. Now we apply our previous theorem
    $$f^{(n)}(z)=\frac{n!}{22 \pi i} \int_{|w-z| = s }\frac{f(w)}{(w-z)^{n+1}} dw$$
  Now let $\gamma_1, \gamma_2$ be freely $\C^2$-homotopic in $D\setminus \{z\}$
    $$=\frac{n!}{2\pi i}\int_{|w-z_0|=r}\frac{f(w)}{(w-z)^{n+1}}dw$$
\end{proof}

\section{14. feb.}
\begin{corollary}
  Let $D \subseteq \C$ open, $f:D\to \C$ a function. TFAE:
    \begin{enumerate}[(i)]
      \item $f$ is complex differentiable (holom.)
      \item $f$ is infinitely many times comp. diff.
      \item $f$ admits a complex antiderivative locally at every point in $D$
      \item $f$ can be written as a complex power series loccally at every point in $D$
      \item $f$ is (real) totally differentiable and satisfies Cauchy-Riemann
      \item $f$ i of class $c^{\infty}$ and satisfies Cauchy-Riemann
    \end{enumerate}
\end{corollary}

\begin{theorem}
  \textbf{Liouville's theorem}
  \newline Every bounded holomorphic function $f:\C \to \C$ is constant.
\end{theorem}
\begin{proof}
  $\exists M \in \mathbb{R}_{> 0}:|f(z)|<M \quad \forall z \in \mathbb{C}$. Furthermore $f(z)=\sum_{n=0}^{\infty} \frac{f^{(n)}(0)}{n !} z^{n} \quad \forall z \in \mathbb{C}$ and $\forall r>0 \quad \forall n \in \mathbb{Z}_{\geq 1}$
  $$\left|\frac{f^{(n)}(0)}{n!}\right|=\left|\frac{1}{2\pi i} \int_{|z|=r}\frac{f(z)}{z^{n+1}}dz\right|$$
  $$\leq \frac{1}{2 \pi} \int_{|z|=r} \left| \frac{f(z)}{z^{n+1}} \right| d z$$
  $$\nleq \frac{1}{2\pi} \frac{M}{r^{n+1}} \cdot 2 \pi r = \frac{M}{r^n}\xrightarrow{r\to \infty}0$$
  $$\Rightarrow f(z)\rvert_{D_r}=\sum_{n=0}^{\infty} \frac{f^{(n)}}{n!} z^n \xrightarrow{r\to \infty}f(0)$$
  \qedhere
\end{proof}
\begin{lemma}
  Let $\emptyset \neq D \subseteq \C$ open, be connected and let $f:D\to \C$ be holomorphic. TFAE:
    \begin{enumerate}[(i)]
      \item $f \equiv 0$ on $D$
      \item $f \rvert_A \equiv 0$ for some set $A\subseteq D$, which has an accumulation point in $D$.
      \item $\exists z_0 \in D: f^{(n)}(z_0)=0 \forall n \in z_{\geq0}$.
    \end{enumerate}
\end{lemma}
\begin{proof}
  $((i)\Rightarrow (ii))$ take $A=D$
  \newline $((ii)\Rightarrow (iii))$ Let $z_0 \in D$ be an accumulation point of $A$. Assume FTSOAC $\exists n \in \Z_{\geq 0}: f^{(n)}(z_0)\neq 0$. Let $n$ be chosen to be minimal with this property
    $\Rightarrow \exists u \subseteq D$ open with $z_0\in u$, such that $\forall z\in u$:
      $$f(z)=\sum_{k=n}^{\infty}\frac{f^{(k)}(z_0)}{k!}(z-z_0)^k \Rightarrow g: u \to \C: z\mapsto \sum_{k=0}^{\infty}\frac{f^{(n+k)}(z_0)}{k!}(z-z_0)^k$$
      is holomorphic on $u$ and $\forall z\in u \setminus\{z_0\} g(z)=\frac{f(z)}{(z-z_0)^n}$
      \newline $\Rightarrow $ the function $h(z_0): D\to \C, z\mapsto
        \begin{cases}
          f(z)/(z-z_0)^n \text{ if }z\neq z_0 \\
          f^{(n)}/n! \text{ if }z=z_0
        \end{cases}
      $.
      \newline Let $(a_m)_{m\in \Z_{\geq 0}}$ be a sequence in $A\setminus \{z_0\}$ converging to $z_0$. We get
        $$0 \neq \frac{f^{(n)}(z_0)}{n!}=h(z_0)=\lim_{m\to \infty}h(a_m)= \lim_{m\to \infty}\frac{f(a_m)}{(a_m-z_0)^n}=0$$
        Contradiction
      \newline $((iii) \Rightarrow (i))$ $S:= \{.z \in D: f^{(n)}(z)=0 \quad \forall n \in \mathbb{Z}_{\geq 0}\}$. Have to show that $S=D$. $D$ connected $\Rightarrow $ enough to show $a) S\neq \emptyset$ $b) S\subseteq D \text{ closed }, c) S\subseteq D \text{ open}$.
        \begin{enumerate}[a)]
          \item $z_0 \in S$
          \item $f^{(n)}$ is continous $\forall n \in \Z_{\geq 0}$. Let $z\in D$ be the point(?) of a sequence $(s_m)_{m\in \Z_{\geq 0}}\in S$. Then $\forall n$, we have
            $$f^{(n)}(z)= f^{(n)}\left(\lim_{m\to \infty}s_m\right) (=cont) \quad lim_{m\to \infty}f^{(n)}(s_m)=\lim_{m\to \infty}0=0$$
            Hence the limit is in $S$ so $\overline{S}=S$, that is $S$ is closed in $D$.
          \item Let $w \in S \Rightarrow \exists$ open neighborhood $u$ of $w$ in $D$ such that
          $$\forall z\in u:f(z)=\sum_{n=0}^{\infty}\frac{f^{(n)}(w)}{n!}(z-w)^n=0$$
          $$\Rightarrow f \equiv 0 \text{ on } U \Rightarrow u \subseteq S$$
          \qedhere
        \end{enumerate}
\end{proof}

\begin{example}
  \begin{enumerate}[(a)]
    \item $f: \C \to \C$ holomorphic, $f(\frac{1}{n})=0 \forall n \in \Z_{>0}, A=\{\frac{1}{n} : n \in \Z_{>0}\}$ $\Rightarrow f \equiv 0$
    \item $g: \C \setminus\{0\} \to \C$ holomorphic, $g(\frac{1}{n})=0 \forall n \in \Z_{>0}$ can choose $g(z)=\sin(\frac{\pi}{2})\not\equiv 0$ but $A=\{\frac{1}{n}: n \in \Z_{Z>0}\}$ has accumulation point -- the problem is that it is in the boundary.
  \end{enumerate}
\end{example}

\begin{theorem}
  \textbf{Identity theorem}
  \newline Let $D\subseteq \C$ open, be connected. If two holom. functions $f,g: D\to \C$ coincide on a set $A\subseteq D$ which has an accumulation point in $D$, then they coincide on all of $D$.
\end{theorem}
\begin{proof}
  Apply the previous lemma to $f-g$.
  \qedhere
\end{proof}

\subsection*{Maximum principle Version 3}
$D\subseteq \C$ open, connected, $f:D\to \C$ holomorphic. If $|f|$ has a local maximum in $D$.

\begin{corollary}
  $D\subseteq \C$ open, connected, $f:D\to \C$ holomorphic and non-constant. If $z_0 \in D$ and $f(z_0)=0$, then there exists an open neighborhood $u\subseteq D$ of $z_0$ such that $z_0$ is the only zero of $f\rvert_u$.
\end{corollary}
\begin{proof}
  Do it yourself.
\end{proof}

\begin{theorem}
  \textbf{Open mapping theorem}
  \newline Let $D\subseteq \C$ open, be connected, $f:D\to \C$ holomorphic and non-constant. Then $f(D)$ is open in $\C$.
\end{theorem}
\begin{proof}
  Let $w_0 \in f(D)$, i.e. $z_0 \in D: f(z_0)=w_0$. Since $f-w_0$ is holomorphic and non-constant ont the connected open set $D$, we find $\varepsilon >0$ such that $\overline{D_{\varepsilon}}(z_0)\subseteq D$ and $z_0$ is the only zero of $f-w_0$ on $\overline{D_{\varepsilon}}(z_0)$. Let $\Delta:=min_{z:|z-z_0|=\varepsilon}|f(z)-w_0| >0$.
  WTS: $D_{\Delta/2}(w_0)\subseteq f(D)$. Let $w\in D_{\Delta/2}(w_0)$ and consider $f-w$. For $z\in D$ with $|z-z_0|=\varepsilon$ we get by triangle inequality
    $$||f(z)-|w|| \geq |f(z)-w_0|-|w_0-w|\geq \Delta -|w-w_0|>\Delta - \frac{\Delta}{2}=\frac{\Delta}{2}$$
    But $$|f(z_0)-w|=|w_0-w|<\frac{\Delta}{2}\Rightarrow \left|f\rvert_{\overline{D_{\varepsilon}(z_0)}}-w\right|$$
    does not attain its minimum on the boundary of $D_{\varepsilon}(z_0)$, which by the maximum principle ($2$nd version) implies that $f\rvert_{\overline{D_{\varepsilon}(z_0)}}-w$ must have a zero $\tilde{z}\in D_{\varepsilon}(z_0)$.
    \qedhere
\end{proof}
