\title{Lecture week 5}
\maketitle
\section{4. feb}
\begin{corollary}
  If $D\subseteq \C$ open, is simply connected (path-connected, every closed $C^0$-path is $C^0$-contractible in $D$), then the complex line integral of holomorphic functions on $D$ only depends on start- and end point.
\end{corollary}
\begin{proof}
  $$0=\int_{\gamma}f(z)dz-\int_{\tilde{\gamma}}f(z)dz$$
\end{proof}

\begin{proposition}
  Let $D\subseteq \C$ open, be simply connected. Then every holomorphic function $f:D\to \C$ admits a complex antiderivative.
\end{proposition}
\begin{proof}
  Exercise.
  Hint:
\begin{itemize}
  \item fix $z_0\in D$.
  \item for $z\in D$ pick arbitrary $C^1$-curve $\gamma_z$ from $z_0$ to $z$
  \item define $F:D\to \C: z\mapsto \int_{\gamma}f(z)dz$
  \item
    $$\frac{F(z+h)-F(z)}{h}=\frac{\int_{\gamma_{z+h}}f(z)dz-\int_{\gamma_z}f(z)dz}{h}=\frac{1}{h}\int_{\text{ straight line segment from z to z+h}}f(z)dz$$
\end{itemize}

\end{proof}

\begin{theorem}\label{CauchyIntForm}
  \textbf{Cauchy Integral Formula}
  \newline Let $D\subseteq \C$ open, $z_0\in D$ and assume $\overline{D_r(z_0)}=\{z\in \C:|z-z_0|\leq r\}\subseteq D$. If $f:D\to \C$ is holomorphic, then we have $\forall z\in D_r(z_0):$
    $$f(z)=\frac{1}{2\pi i}=\int_{|w-z_0|=r}\frac{f(w)}{w-z}dw$$
    In particular, the values of $f$ inside the disc are uniquely determined by the values in the boundary of the disc.
\end{theorem}

\section{7.feb}
\begin{theorem}
  \textbf{Cauchy Integral Formula}
  \newline Let $D\subseteq \C^n, z_0\in D, r\in \R_{>0}, \overline{D_r(z_0)}\subseteq D$.
  \newline If $f:D\to \C$ is holomorphic, then, $\forall z\in D_r(z_0)$
    $$f(z)=\frac{1}{2\pi i} \int_{|w-z_0|=r}\frac{f(w)}{w-z}dw.$$
  If $\gamma$ is a closed $C^1$ curve homotopic to $|w-z_0|=r$ in $D\setminus \{z\}$, then $f(z)=\frac{1}{2\pi \varepsilon}\int_{\gamma}\frac{f(w)}{w-z}dw$.
\end{theorem}
\begin{proof}
  Fix a path $z\in D_r(z_0)$. Then $w\mapsto \frac{f(w)}{w-z}$ is holomorphic in $D\setminus\{z\}$. So by invariance of complex line integral under homotopies, we have for
    $0<\varepsilon<< 1: \overline{D_{\varepsilon}(z)}\subseteq D_r(z_0)$ and
    $$\frac{1}{2\pi i}\int_{|w-z_0|=r}\frac{f(w)}{w-z}dw=\frac{1}{2\pi i}\int_{|w-z|=\varepsilon}\frac{f(w)}{w-z}dw$$
    $$=\frac{1}{2\pi i}\int_{|w-z|=\varepsilon}\frac{f(w)-f(z)}{w-z}dw+\frac{1}{2\pi i}\int_{|w-z|=\varepsilon}\frac{f(z)}{w-z}dw$$
    Let's look at just $\frac{1}{2\pi i}\int_{|w-z|=\varepsilon}\frac{f(z)}{w-z}dw$
    $$=\frac{1}{2\pi i}f(z) \cdot\int_{|w-z|=\varepsilon}\frac{1}{w-z}dw$$
    $$=\frac{1}{2\pi i}f(z) \cdot \int_{0}^{2\pi }\frac{1}{\varepsilon \cdot e^{i\theta}} \cdot \varepsilon \cdot i \cdot e^{i\theta}d\theta$$
    $$=f(z)$$
    Define
    $$g:D\to\C:w\mapsto
\begin{cases}
  \frac{f(w)-f(z)}{w-z} \text{ if }w\neq z \\
  f'(z) \text{ if } w=z
\end{cases}
    $$
    $g$ is continuous $\Rightarrow \exists M\in \R_{>0}:|g|\leq M$ on $\overline{D_r(z_0)}$.
    $$\Rightarrow \left|f(z)-\frac{1}{2\pi i} \int_{|w-z_0|=r}\frac{f(w)}{w-z}dw  \right|$$
    $$\left|\frac{1}{2\pi i} \int_{|w-z|=\varepsilon}g(w)dw\right|\leq \frac{1}{2\pi i}(2\pi \varepsilon) \cdot max_{w:|w-z|=\varepsilon}|g(w)|$$
    $$\leq \varepsilon \cdot M \to 0$$
    as $\varepsilon \to 0$.
\end{proof}

$$D=\C\setminus\{0\}, f:D\to \C: w\mapsto \frac{1}{w}$$
Given $z\in D_1(0)\cap D=D_1(0)\setminus\{0\}$, do we have
  $$\frac{1}{z}= \frac{1}{2\pi i}\int_{|w|=1}\frac{f(w)}{w-z}dw??$$
  $$|\frac{1}{z}|\xrightarrow{z\to 0}\infty,$$
  but
  $$\left|\frac{1}{2\pi i}\int_{|w|=1}\frac{f(w)}{w-z}dw\right|\leq \frac{1}{2\pi } \left| \int_{0}^{2\pi}\frac{1}{e^{i\theta} \cdot (e^{i\theta}-z)} \cdot i \cdot e^{i\theta}d\theta \right|$$
  $$\leq \frac{1}{2\pi } \int_{0}^{2\pi}\frac{1}{|e^{i\theta}-z|}d\theta \leq (0<z<1) \frac{1}{2\pi } \cdot \int_{0}^{2\pi}\frac{1}{|e^{i\theta}|-|z|}d\theta$$
  $$=\frac{1}{2\pi}\frac{1}{1-|z|}d\theta$$
  $$=\frac{1}{1-|z|}\xrightarrow{z\to 0}1$$

\begin{lemma}
  Let $D\subseteq \C$ open, $f:D\to \C$ holomorphic, $\overline{D_r(z_0)}\subseteq D$.
  \newline Then $f(z_0)=\frac{1}{2\pi}\int_{0}^{2\pi}f(z_0+r \cdot e^{i\theta})d\theta$.
\end{lemma}
\begin{proof}
  $$f(z_0)=\frac{1}{2\pi i} \int_{0}^{2\pi }\frac{f(z_0+r e^{i\theta})}{r \cdot e^{i\theta}} \cdot r \cdot i \cdot e^{i\theta}d\theta$$
  $$=\frac{1}{2\pi}\int_{0}^{2\pi}f(z_0+r \cdot e^{i\theta})d\theta$$
\end{proof}

\begin{proposition}
  \subsubsection*{Maximum Principle, 1st version}
  $D\subseteq \C$ open, $f:D\to \C$ holomorphic. Further, let $z_0\in D$ be a local maximum of $|f|$. Then $f$ is constant in a neighborhood of $z_0$.
\end{proposition}
\begin{proof}
  $\exists \varepsilon >0: $ the restriction of $|f|$ to $\overline{D_{\varepsilon}|z_0|}$ has a global maximum in $z_0$. Hence, $\forall 0<r\leq \varepsilon:$
    $$|f(z_0)|=\left| \frac{1}{2\pi}\int_{0}^{2\pi}f(z_0+r \cdot e^{i\theta})d\theta \right|\leq \frac{1}{2\pi} \int_{0}^{2\pi}|f(z_0+r \cdot e^{i\theta})|d\theta$$
    $$\leq \frac{1}{2\pi} \int_{0}^{2\pi}|f(z_0)|d\theta=|f(z_0)|$$
    $$\Rightarrow 0=\int_{0}^{2\pi}\left(|f(z_0)|-|f(z_0+r \cdot e^{i\theta})|\right)d\theta$$
    So, by continuity:
      $$|f(z_0)|=|f(z_0 + r \cdot e^{i\theta})| \forall \theta \in[,0,2\pi].$$
      $$\Rightarrow f \text{ constant on }\overline{D_{\varepsilon}(z_0)}.$$
    Exercise Week $3$ $\Rightarrow f$ constant on $D_{\varepsilon}(z_0)$, since the latter set is open, connected.
    \qedhere
\end{proof}

\begin{corollary}
  $D\subseteq \C$ open, $f:D\to \C$ holomorphic. If $|f|$ has a local minimum in $z_0$, then $f(z_0)=0$ or $f$ is constant in a neighborhood of $z_0$.
\end{corollary}
\begin{proof}
  If $f(z_0)\neq 0$, then $\frac{1}{f}$ is holomorphic in a neighborhood of $z_0$.
  \qedhere
\end{proof}

\begin{proposition}
  \subsubsection*{Maximum Principle, 2nd version}
  $D\subseteq \C$ open, $D$ bounded, $f:D\to \C$ is continuous on $\overline{D}$, holomorphic on $D$. Then $\exists p \in \partial D$ s.t.
    $$|f(p)|=max_{z\in \overline{D}}|f(z)|.$$
  If furhtermore $f$ has no zero on $D$, then $\exists q\in \partial D$, s.t. $|f(q)|=min_{z\in \overline{D}}|f(z)|$.
\end{proposition}
\begin{proof}
  $\overline{D}$ is compact, $|f$ is continuous $\Rightarrow M=max_{z\in \overline{D}}|f(z)|\in \R_{\geq 0}$ well-defined. Let $K:=|f|^{-1}(\{M\})\subseteq \overline{D}$ closed. So $K$ is compact and non-empty. If $K\cap \partial D \neq \emptyset $, we are done, so assume FTSOAC that $K\cap \partial D = \emptyset$. Then $K\subseteq D$.
  Let $a\in \partial_{\C}K\subseteq K\subseteq D$.
  \newline $a\in K\subseteq D$, so $|f|$ has a local maximum at $a\in D$.
  \newline $\Rightarrow$ constant in a neighborhood in $\C$ of $a$ $\Rightarrow $ an open (in $\C$) neighborhood of $a$ is contained in $K$
  \newline $a\notin \partial_{\C}K$. Contradiction.
\end{proof}
