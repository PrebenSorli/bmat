\title{Lecture Week 7 Complex Analysis}
\maketitle
\section{18. feb.}
\begin{definition}
  A Laurent SEries centered at $z_0\in \C$ is a formal expression of the form
    $$f(z)=\sum_{n=-\infty}^{\infty}a_n(z-z_0)^n:=\sum_{n=1}^{\infty}a_{-n}(z-z_0)^{-n}+\sum_{n=0}^{\infty}A_{n}(z-z_0)^n$$
    $$f^{-}(z)= \sum_{n=1}^{\infty}a_{-n}(z-z_0)^{-n}, \quad f^{+}= \sum_{n=0}^{\infty}a_n(z-z_0)^{n}$$
    Spør Markus
    \newline The series is said to converge for some $z\in \C$ provided both $f^{+}(z)$ and $f^{-}(z)$ both converge.
\end{definition}

\begin{remark}
  $\lim_{N\to \infty}\sum_{n=-N}^{N}a_n(z-z_0)^n$ exists does not imply $f(z)$ converges.
  \newline Because look at $a_n =
      \begin{cases}
          0 \text{ if } n = 0 \\
          (-1)^n \text{ if } n >0 \\
          (-1)^{n+1} \text{ if } n<0
      \end{cases}$, then
        $$ f^{+}(z)=-z+z^{2}-z^{3} \quad, f^{-}(z) =\frac{1}{2}-\frac{1}{z^{2}}+\frac{1}{z^{3}}-\ldots $$
      With $z=1$ this celarly doesn't converge for $f^{+} (z)$ and same for $f^{-}(z)$, but look at the limit: $\sum_{n=-N}^{N}a_n \cdot 1^n=0 \forall N$, so $\lim_{N\to \infty}\sum_{n=-N}^{N} A_n \cdot 1^n = 0$.
\end{remark}

\begin{remark}
  Let $f(z)=\sum_{n=-\infty}^{\infty}a_n(z-z_0)^n$ be a Laurent series. Set
    $$r:= \lim_{n\to \infty}\sup\sqrt[n]{|a_{-n}}\in \R_{\geq 0} \cup \{+\infty\}$$
    $$R:= \lim_{n\to \infty} \sup \frac{1}{\sqrt[n]{|a_n|}}\in \R_{\geq 0}\cup \{ + \infty \}$$
    $$U:=\{z\in \C : r < |z-z_0 | < R\}$$
    then
      \begin{enumerate}[(a)]
        \item $f(z)$ converges absolutely if $z\in u$
        \item $f(z)$ diverges  if $|z-z_0|<r$ pr $>R$
        \item The partial sums of $f^{+}$ and $f^{-}$ converge uniformly on compact subsets of $u$. $u$ is called the \emph{annulus of convergence }of $f$.
        \item $f$ defines a holomorphic function on $u$, which can be differnetiated term by term. Teh resulting Laurent series has the same annulus of convergence.
      \end{enumerate}
\end{remark}
\begin{proof}
  If $g_1(w)= \sum_{n=0}^{\infty}a_n w^n, g_2(w)=\sum_{n=1}=a_{-n}w^n$, then radius of convergence of $g_1$ is $R$, the radius of convergence of $g_2$ is $\frac{1}{r}$ (root test, i.e. Cauchy-Hadamard Formula). The claims follows by applying the corresponding results for power series.
\end{proof}

\subsection*{Laurent Series expansion of holomorphic functions}
Let $D\subseteq \C$ open, $f:D\to \C$ holomorphic. Furthermore let $r,R\in \R_{\geq0} \cup \{+\infty\}, r <R$, such that
  $$u: ) \{ z\in \C : r< |z-z_0| < R \} \subseteq D.$$
Then
  \begin{enumerate}[(a)]
    \item On $u$, the function $f$ is given by $f(z)=\sum_{n=-\infty}^{\infty}a_n(z-z_0)^n \forall z\in u$.
    \item The coefficients are uniquely determined: they satisfy $\forall n \in \Z: a_n = \frac{1}{2\pi i}\int_{|z-z_0|=\rho}\frac{f(z)}{(z-z_0)^{n+1}}$ for some (and hence any by homotopy) $\rho$ with $r < \rho <R$.
  \end{enumerate}

\subsection*{Proof sketch:}
$$f(z)= \frac{1}{2\pi i} \int_{\gamma}\frac{f(w)}{w-z }dw$$
By figure we get
\newline Sett inn figur
  $$\int_{\gamma}\frac{f(w)}{w-z}dw= \int_{|w-z_0|=r_2}\frac{f(w)}{w-z}- \int_{|w-z_0|=r_1}\frac{f(w)}{w-z}$$
  Let the two integrals be denoted $\RomanNumeralCaps{1}$ and $\RomanNumeralCaps{2}$ respecitvely.
\newline Now we use the power series trick as
  $$\left|\frac{z-z_0}{w-z_0}\right|<1 \text{ and } \left|\frac{w-z_0}{z-z_0}\right|<1$$
For $\RomanNumeralCaps{1}$:
  $$\frac{f(w)}{w-z}= \frac{f(w)}{w-z_0} \cdot \frac{1}{1-\frac{w-z_0}{z-z_0}} = \frac{f(w)}{w-z_0}\sum_{n=0}^{\infty}\left(\frac{z-z_0}{w-z_0}\right)^n$$
For $\RomanNumeralCaps{2}$:
  $$\frac{f(w)}{w-z}= \frac{f(w)}{w-z_0} \cdot \frac{1}{1-\frac{w-z_0}{z-z_0}} = \frac{f(w)}{w-z_0}\sum_{n=0}^{\infty}\left(\frac{w-z_0}{z-z_0}\right)^n$$
The rest of the proof is analogous to the way we proved this for power series. This proves existence, now we prove uniqueness. Assume that $f(z)= \sum_{n=-\infty}^{\infty}a_n(z-z_0)^n$ is one such representation of $f$ on $u$. Then  for $n \in \Z, \rho \in (r,R)$:
  $$\frac{1}{2\pi i}\int_{|z-z_0|=\rho}\frac{f(z)}{(z-z_0)^{n+1}}dz = \frac{1}{2 \pi i}\sum_{k=-\infty}^{\infty}\int_{|z-z_0|=\rho}a_k \frac{(z-z_0)^k}{(z-z_0)^{n+1}}dz$$
  $$ \frac{1}{2\pi i }\sum_{k=-\infty}^{\infty}a_k \int_{|z-z_0|=\rho}(z-z_=)^{k-n-1}dz$$
\begin{recall}
  $$\int_{|w|=1}w^n dw =
      \begin{cases}
          0 \text{ if } n \neq -1 \\
          2 \pi i \text{ if } n = -1
      \end{cases}$$
\end{recall}
By recall we have
  $$\int_{|z-z_0|=\rho}(z-z_0)^{k-n-1}dz =
    \begin{cases}
      2 \pi i \text{ if }k-n-1=-1
      0 \text{ otherwise}
    \end{cases}
  $$
Hence
  $$ \frac{1}{2\pi i}\sum_{k=-\infty}^{\infty}a_k \int_{|z-z_0|=\rho}(z-z_0)^{k-n-1}dz = \frac{1}{2\pi i}a_n \cdot 2\pi i = a_n $$
\qedhere

\begin{remark}
  (Notation as above)
  \newline If $D_{R}(z_0)\subseteq D$, then the Laurent expansion of $f$ around $z_0$ and the Taylor expansion of $f$ around $z_0$ coincide.
\end{remark}
\begin{example}
  $f: \C \setminus \{-1,1,-i,i\} \to \C: z \mapsto \frac{1}{1-z^n}, z_0=1$
  $$u_1 = \{z\in \C : 0 < |z-1| < \sqrt{2}\}$$
  $$u_2 = \{z\in \C : \sqrt{2} < |z-1| < 2 \}$$
  $$u_3 = \{z \in \C : 2 < |z.1 | \}$$
Hence $f$ has three Laurent expansions as we need to partition them to avoid the singularities.
\end{example}

\begin{definition}
  $D\subseteq \C$ open, $f: D\to \C$ holomorphic, $z\in \C \setminus D$.
    \begin{enumerate}
      \item If $\exists \varepsilon >0 : D_{\varepsilon} \setminus \{z_0\} \subseteq D$, then $z_0$ is calle  an \emph{isolated singulairty}.
      \item If $z_0$ is an isolated singularity and $f$ extends holomorphically to $D\cup \{ z_0 \}$, then $z_0$ is called a \emph{removable singularity}.
    \end{enumerate}
\end{definition}

\begin{example}
  $f,y : \C\setminus\{0,1\} \to \C, z\xrightarrow{ f} \frac{1}{z}, z \xrightarrow{g} \frac{e^z-1}{z}$
  \newline $0$ and $1$ are isolated singularities of $f$ and $g$. $1$ is a removable singularity of $f$ and $g$. $0$ is not a removable singularity of $f$.
    $$0 \neq g(z)=\frac{1}{z}\left(-1 + \sum_{k=0}^{\infty}\frac{z^k}{k!} \right)= \frac{1}{z}\sum_{k=0}^{\infty} \frac{x^{k+1}}{(k+1)!}$$
    $$= \sum_{k=0}^{\infty}\frac{1}{(k+1)!}z^k \Rightarrow 0 \text{ is a removable singularity of }g.$$
\end{example}

\begin{definition}
  $f: D \to \C$ holomorphic, $z_0 \in \C$ ($z_0 \in D$ or $z_0 \in \C\setminus D$ isolated singularity of f). So for $0 < \varepsilon < < 1$ (small enough), we have $D_{\varepsilon}(z_0)\setminus \{z_0\}\subeteq D$, i.e. on this set $f$ admits a uniquely determined expansion
    $$f(z)= \sum_{n=-\infty}^{\infty}a_n(z-z_0)^n, z \in D_{\varepsilon}(z_0)\setminus\{z_0\}$$
    which is independent from $0 < \varepsilon <<1$. If $f \not\equiv 0$ on $D_{\varepsilon }(z_0)\setminus\{z_0\}$, then we call
    $$ord_{z_0}f: =
      \begin{cases}
        min\{n\in \Z : a_n \neq 0 \text{, if this set is bounded from below} \\
        - \infty \text{, otherwise}
      \end{cases}$$
    the \emph{order} of $f$ in $z_0$. Write $m=ord_{z_0}f \in \Z \cup \{ -\infty\}$.
      \begin{enumerate}[(a)]
        \item If $m>0$ then $z_0$ is called a zero of order $m$ (of $f$)
        \item If $m<0$, then $z_0$ is called a pole of order $-m, m\neq -\infty$ of $f$.
        \item If $m= - \infty$, then $z_0$ is calledc an essential singularity of $f$
      \end{enumerate}
\end{definition}

\section{21. feb}
\begin{lemma}
  Let $f:D \to \C$ holom., $z_0\in \C$ s.t. $z_0\in \C$ or $z_0\in \C\setminus D$ isolated sing. of $f$. Assume $f \not\equiv 0$ on $D_{\varepsilon}(z_0)\setminus\{z_0\}$ for $0< \varepsilon <<1$ and assume $ord_{z_0}f \neq - \infty$. Then $ord_{z_0}f$
  is the uniquely determined integer $m\in \Z$, s.t. $\exists g:D \cup \{z_0\} \to \C$ holom. with $g(z_0)\neq 0$ and $\forall z\in D\setminus\{z_0\} f(z)=(z-z_0)^m \cdot g(z)$.
\end{lemma}
\begin{corollary}
  \begin{enumerate}[(a)]
    \item $f,f_1,f_2 : D\to \C$ holom., $z_0$ isolated sing. of $f,f_1,f_2$. Assume $f,f_1,f_2 \not\equiv 0$ on $D_{\varepsilon}(z_0)\setminus \{z_0\}$ for $0<\varepsilon <<1$ and $ord_{z_0}f$, and $ord_{z_0}f_1$, $ord_{z_0}f_2 \neq -\infty$.
    Then
      $$ord_{z_0}(f_1 \cdot f_2)= ord_{z_0}f_1 + ord_{z_0}f_2 \text{ and } ord_{z_0}(\frac{1}{f})=-ord_{z_0}f$$
    \item $f: D\to\C$ holom., $z_0$ isolated sing. of $f$, $f\not\equiv 0$ on $D_{\varepsilon}(z_0)\setminus \{z_0\}$ for $0<\varepsilon <<1$. Then $z_0$ is a \emph{removable} singularity $\Leftrightarrow ord_{z_0}f \geq 0$
  \end{enumerate}
\end{corollary}

\begin{theorem}
  \textbf{Riemann Removable Singularity Theorem}
  $D\subseteq \C$ open, $f: D\to \C$ holom., $z_0\in \C\setminus D$ is of singularity of $f$. If $\exists $ neigbourhood $u$ of $z_0$ in $\C$, s.t. $u\setminus\{z_0\}\subseteq D$ and $f$ is bounded on $u\setminus\{z_0\}$, then $z_0$ is a removable singularity of $f$.
\end{theorem}
\begin{proof}
  Let $M\in \R_{>0}$ and $\varepsilon >0$ s.t.: $D_{\varepsilon_0}(z_0)\setminus \{z_0\}\subseteq D$ and $|f|<M$ on $D_{\varepsilon_0}(z_0)\setminus \{z_0\}$. Then $\forall z\in D_{\varepsilon_0}(z_0)\setminus \{z_0\}$
    $$f(z)=\sum_{n=-\infty}^{\infty}a_n \cdot(z-z_0)^n, \text{ where } \forall n\in \Z$$
    $$\forall 0<\varepsilon<\varepsilon_0: a_n=\frac{1}{2\pi i} \int_{|w-z_0|=\varepsilon}\frac{f(w)}{(w-z_0)^{n+1}}dw$$
    $$|a_n|\leq \frac{1}{2\pi } \cdot 2\pi \varepsilon \frac{M}{\varepsilon^{n+1}}=M \cdot \varepsilon^{-n} \to 0 (\varepsilon \to 0 (n<0))$$
    $$\Rightarrow f(z)=\sum_{n=0}^{\infty}a_n(z-z_0)^n \forall z\in D_{\varepsilon}(z_0)\setminus \{z_0\}$$
    $$\Rightarrow f \text{ extends holomorphically to }z_0 \text{ via }f(z_0)=a_0.$$
\end{proof}
\subsection*{Behaviour near isolated singularity}
$D\subseteq \C$ open, $f: D\to \C$ holom., $z_0\in \C\setminus D$ isolated singularity, i.e. $D_{\varepsilon_0}(z_0)\setminus \{z_0\}\subseteq D$ for some $\varepsilon >0$.
  \begin{itemize}
    \item if $f \equiv 0$ on $D_{\varepsilon_0}(z_0)\setminus \{z_0\}$, then $z_0$ is a removable sing.
    \item if $f \not \equiv 0$ on $D_{\varepsilon_0}(z_0)\setminus \{z_0\}$ and $ord_{z_0}f\geq0$, then $z_0$ is a removabl sing.
    \item if $f\not \equiv 0$ on $D_{\varepsilon_0}(z_0)\setminus \{z_0\}$ and $ord_{z_0}f<0$, but $\neq -\infty$, then
      $$|f(z)| \to \infty \text{ as }z \to z_0$$
      \item (Casarati-Weierstrass)

       if $f \not\equiv 0$ on $D_{\varepsilon_0}(z_0)\setminus \{z_0\}$ and $ord_{z_0}f=-\infty$, then $\forall u\subseteq \C$ open with $z_0\in u$, $u\setminus\{z_0\} \subseteq D$, we have: $f(u\setminus\{z_0\})$ is dense in $\C$.
  \end{itemize}
\begin{proof}
  Assume FTSOAC $\exists u\subseteq \C$ open, with $z_0\in u$ and $u\setminus\{z_0\}\subseteq D$, s.t.
    $$f(u\setminus\{z_0\})\cap D_\delta(w_0)=\emptyset \text{ for some }\delta>0, w_0\in \C, i.e.$$
    $$|f(z)-w_0|\geq \delta \forall z\in u\setminus\{z_0\}. \Rightarrow g:u\setminus\{z_0\}\to \C, z\mapsto \frac{1}{f(z)-w_0}$$
    is well-defined, holom. bounded, so by RRST, $g$ extends holomorphically to
      $$g: u\to \C. \text{ On }u\setminus\{z_0\}: f(z)-w_0=\frac{1}{g(z)}$$
      $$\Rightarrow \Z \ni -ord_{z_0}g=ord_{z_0}\frac{1}{g}=ord_{z_0}(f-w_0)=ord_{z_0}f=-\infty \Rightarrow -\infty\in \Z$$
      Contradiction.
      Above we use $g:u\to \C$ holomorphic, $g$ nowhere $0$ on $u\setminus\{z_0\}$.
      \qedhere
\end{proof}
\begin{definition}
  Let $z_0\in\C$ and let $\gamma$ be a closed $pwC^1$ curve in $\C\setminus\{z_0\}.$ Then
    $$ind_{z_0}\gamma := \frac{1}{2 \pi i} \cdot \int_{\gamma}\frac{1}{z-z_0}dz$$
  is an integer. It is called the \emph{index} or \emph{winding number of }$\gamma$ around $z_0$.
  \newline see exercises week 4
\end{definition}
\begin{remark}
  Let $\gamma:[a,b]\to \C\setminus\{0\}$ be a closed $pwC^1$ curve and assume that
  \newline We choose $arg\in(-\pi , \pi]$
    \begin{enumerate}[(a)]
      \item $T=\{t\in [a,b]: \gamma(t)\in \R_{<0}$ is a finite set.
      \item $a,b \notin T$.
      \item $T=T^+ \cup(dot) T^-$, where
        \begin{itemize}
          \item $T^+=\{t\in T: \lim_{s\nearrow t}arg \gamma(s)=\pi, \lim_{s\searrow t}arg \gamma (s)=-\pi\}$
          \item $T^-=\{t\in T: \lim_{s\nearrow t}arg \gamma(s)=-\pi, \lim_{s\searrow t}arg \gamma (s)=\pi\}$
        \end{itemize}
    \end{enumerate}
    Then $ind_0 \gamma=card T^+-cardT^-$.
\end{remark}
