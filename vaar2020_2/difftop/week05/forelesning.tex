\title{Lecture week 5 DiffTop}
\maketitle
\section{4. feb}
\subsection*{Local properties of smooth maps}
$f: X \to Y$
\newline $dimX\geq dimY$.
\begin{definition}
  A map $f: X \to Y$ is a submersion at $x\in X$ if $Df_x:T_xX \to T_{f(x)}Y$ is surjective.
\end{definition}
\begin{example}
  Canonical submersion is a projection map
    $$can:\R^{n+k}\to \R^n$$
    $$(x_1,\dots,x_m,x_{n+1},\dots,x_{n+k})\mapsto(x_1,\dots,x_n)$$
  is a submersion at all points.
    $$D can_x=
    \begin{bmatrix}
      I_n | 0
    \end{bmatrix}$$
    $n\times(n+k)$.
\end{example}

\begin{proposition}
  Let $f: X \to Y$ be a submersion at $x\in X$ , then there are coordinate systems about $x\in X$ and $f(x)\in Y$ so that the map $f$ appears to be the canonical submersion in coordinates.
\end{proposition}
\begin{proof}
  Find some local coordinates
  \newline Diagram 1
  \newline Assumption that $Df_x$ is surjective implies $Dh_0$ is surjective.
  \newline We can find a matrix $Q$ so that
    $$dh_0 \cdot Q =
    \begin{bmatrix}
      I_n | 0
    \end{bmatrix}$$
    with $Q$, we can modify our coordinates about $x\in X$. %Diagram 1.2
    In the new coordinates, $f$ appears to be $h\circ Q$, and so
      $$d \left(h\circ Q\right)_0=dh_0 \cdot Q =
      \begin{bmatrix}
        I_n | 0
      \end{bmatrix}$$
    Assume we've done this, redefine our charts so
    \newline diagram 2
      $$dh_0=
      \begin{bmatrix}
        I_n | 0
      \end{bmatrix}$$
      Now define a new map
        $$G:\tilde{u}\to \tilde{v}\times \R^k$$
        $$\tilde{u}\subseteq \R^{n+k}, \tilde{v}\subseteq \R^{n+k}$$
        $$G(x)=\left(h(x),x_{n+1},\dots,x_{n+k}\right)$$
        $$dG_0
        \begin{bmatrix}
            dh_0 | & 0 \\
            0 | & I_k
        \end{bmatrix}=
        \begin{bmatrix}
          I_n | & 0 \\
          0 | & I_k
        \end{bmatrix}=I_{n+k}$$
      Observe $G$ is a local diffeo at $0$ because $dG_0=I_{n+k}$ by I.F.T. So we can pick $u'\subseteq \tilde{u}$ and $w\subseteq \tilde{v}\times \R^k$ where $G:u'\xrightarrow{\simeq}w$. Use coordinates $\varphi \circ G ^{-1}$ about $x\in X$, then $f$ is just canonical submersion in coordinates.
\end{proof}

\begin{proposition}
  If $f: X \to Y$ is a submersion at $x\in X$, then $f$ is a submersion in an open neighborhood around $x$.
\end{proposition}
\begin{proof}
  Find good coordinate systems:
  \newline Diagram 3
  \newline for any point $u\in \mathcal{U}$,
    $$Df_u=D\psi_{can(\varphi (u))}\circ Dcan_{\varphi (u)}\circ D \varphi ^{-1}_{\varphi (u)}$$
    isos plus just $[I_n | 0]$ so surjective, $\Rightarrow Df_u$ is surjective.
\end{proof}

\begin{proposition}
  If $f: X \to Y$ is smooth, and $f(x)=y$, and $f$ is submersion at $x\in X$, then $f ^{-1}(y)$ is locally euclidean.
\end{proposition}
\begin{proof}
  Pick coords so $f$ is the canonical submersion at $x\in X$.
  \newline diagram 4
  \newline consider $f ^{-1}(y)\cap \mathcal{U}$ open relative to $f ^{-1}(y)$. I claim this is diffeo to open subset of $\R^k$.
    $$f ^{-1}(y)\cap \mathcal{U}\simeq \varphi ^{-1}(f ^{-1}(y))$$
    $$=can ^{-1}(\psi ^{-1}(y))=can ^{-1}(0)=\{x\in \tilde{\mathcal{U}}|x_1=0,x_2=0,\dots,x_n=0\}$$
    first diffeo is diffeo via $\varphi$.
  \newline picture 1
  \newline to get coords for $can ^{-1}(0)$, use another projection map.
    $$\pi:\R^{n+k}\to \R^k$$
    $$(x_1,\dots,x_{n+k})\mapsto(x_{n+1},\dots,x_{n+k})$$
  if we look at
    $$can ^{-1}(0)=\{x\in \tilde{\mathcal{U}}|x_1=0,x_2=0,\dots,x_n=0\}$$
    $$can ^{-1}(0) \xrightarrow{\pi}\R^k$$
  we get $W=\pi(can ^{-1}(0))$ and
  $$\{x\in \tilde{\mathcal{U}}|x_1=0,x_2=0,\dots,x_n=0\}\simeq W \subseteq \R^k \text{ open}.$$
    $$can ^{-1}(0) \xrightarrow{\pi, \simeq}W$$
  is a diffeo.
\end{proof}
\begin{example}
  $$f:\R^{n+1}\to \R$$
  $$f(x)=x_1^2+x_2^2+\dots +x_{n+1}^2$$
  $$f ^{-1}(1)=\{x\in \R^{n+1} | \sum x_i^2=1\}=S^n.$$
\end{example}

\begin{definition}
  Let $f: X \to Y$ be smooth. $y\in Y$ is a \emph{regular value} of $f$ if:
    $$\forall x\in f ^{-1}(y), Df_x \text{ is surjective}$$
    i.e., $f$ is a submersion at $x$.
\end{definition}
\begin{theorem}
  Let $f: X \to Y$ smooth, suppose $X, Y$ connected, $dim X=n, dim Y =k$. If $y\in Y$ is a regular value of $f$, $f ^{-1}(y)$ is a manifold. If $f ^{-1}(y)\neq \emptyset$, then $dim f ^{-1}(y)=dim X-dim Y=n-k$.
\end{theorem}
\begin{proof}
  $$\forall x\in f ^{-1}(y)$$
  as $f$ is a submersion at $x$, we know $f ^{-1}(y)$ is locally euclidean at $x$. So $f ^{-1}(y)$ is a manifold. And $dim f ^{-1}(y)$ is obtained from previous prop too.
\end{proof}
\begin{example}
  $S^1$ is a smoth manifold.
    $$f:\R^{n+1}\to \R$$
    $$f(x)=\sum_{i=1}^{n+1}x_i^2$$
    $$df_x=[2x_1,2x_2,\dots,2x_{n+1}]$$
  is surjective when?
  \newline $x\neq 0$
  \newline Hence $f ^{-1}(y)\subseteq \R^{n+1}$ is a manifold as long as $0\notin f ^{-1}(y)$, i.e. for all $y\in \R\setminus\{0\}$.
    $$S^n=f{-1}(1).$$
    $$\emptyset=f^{-1}(-1)$$
  Regular values of $f$ are $\R\setminus\{0\}$.
\end{example}
Which manifolds can be obtained as the preimage of a regular value?
\newline Specifically
  \begin{enumerate}[(1)]
    \item $$f:\R^n\to \R^k$$
    When is a manifold $X \subseteq \R^n $ obtained as $X=f^{-1}(0)$ for $0$ a regular value of $f$?
    \newline One way to think of this is
      $$f=(f_1,\dots,f_k)$$
    for $f_i:\R^n\to \R$ is a ``smooth condition'', then $f_i ^{-1}(0)\subseteq \R^n $ are points with $1$ less degree of freedom.
    \newline So $f ^{-1}(0)=f_1 ^{-1}(0)\cap f_2 ^{-1}(0) \cap \cdots \cap f_k ^{-1}(0)$.
    \item One modification, which submanifolds of a manifold $X$ are obtained from $f: X\to \R^k$
    $f ^{-1}(0)$, $0$ a regular value.
    \newline If we take $k=0$. Consider $f:X\to \R^0$, $\R^0=\{0\}, T_0\R^0=\{0\}$.
      $$Df_x:T_xX \to T_0\R^0$$
    is surjective always, so $f ^{-1}(0)=X$.
    \item ``Partial answers''
    \newline GP Partial converse 1 not interesting.
    \item
      \begin{proposition}
        Let $Z\subseteq X$ be a submanifold, then there is a smooth function $f: X \to \R^k$ so that $Z$ is locally $f ^{-1}(0)$, i.e., for $x\in Z$, we can find $u\subseteq X$ open and $f: X \to \R^k$, so that $f ^{-1}(0)\cap u=z\cap u$.
      \end{proposition}
      \begin{proof}
        $Z\xrightarrow{i}X, x\mapsto x$ is an immersion.
        \newline We have
        \newline Diagram 5
        \newline Check that
          $$f ^{-1}(0)=Z\cap V$$
      \end{proof}
      \begin{example}
        $\{(0,0,1)\}\subseteq S^2$, there is no smooth function $f: S^2 \to \R^2$ so that $f ^{-1}(0)=Z$, $0$ a regular value.
        \newline Proof using degree theory.
      \end{example}
  \end{enumerate}

\begin{proposition}
  $f: X \to Y$ smooth, $Z=f ^{-1}(y)$, $y$ a regular value, then $T_xZ=ker(Df_x:T_xX\to T_yY)$.
\end{proposition}
\begin{example}
  $S^n=f ^{-1}(1)$
    $$T_xS^n=ker(df_x=[2x_1,\dots,2x_{n+1}])=\{(x,v)| x\cdot v=0\}$$
\end{example}

$$\C\simeq \R^2$$
$$x+iy\leftrightarrow (x,y)$$
We can think of $S^2$ as $\C$ (or $\R^2$) with ``point at infinity''.
\newline Formula for $\varphi_N(x,y)$
  $$l(t)=
\begin{bmatrix}
  0\\
  0\\
  1
\end{bmatrix}+
t \left(
\begin{bmatrix}
  x\\
  y\\
  0
\end{bmatrix}
-
\begin{bmatrix}
  0\\
  0\\
  1
\end{bmatrix}\right)
  $$
$$=
\begin{bmatrix}
  tx \\
  ty\\
  1-t
\end{bmatrix}=\varphi_N(x,y)
$$
when $t$ is such that $\in S^2$.
\newline Need that $$(tx)^2+(ty)^2+(1-t)^2=1$$
$$t^2(x^2+y^2)+1-2t+t^2=1$$
$$t \left(t(x^2+y^2+1)-2\right)=0$$
need $t(x^2+y^2+1)=2$
$$t=\frac{2}{x^2+y^2-1}$$
$$\varphi _N(x,y)=\frac{1}{x^2+y^2+1} \cdot
\begin{bmatrix}
  2x \\
  2y\\
  x^2+y^2-1
\end{bmatrix}$$
$$\varphi _N ^{-1}\left(
\begin{bmatrix}
  x\\
  y\\
  z
\end{bmatrix}
\right)=\frac{1}{1-z}
\begin{bmatrix}
    x\\
    y
\end{bmatrix}.$$
$$\varphi _S,\varphi _S ^{-1}$$
$$\varphi _S ^{-1} \circ \varphi _N, \varphi_N ^{-1} \circ \varphi _S$$

\section{6. feb}
Cauchy-Riemann says
  $$P(x+iy)=u(x,y)+iv(x,y)$$
is holomorphic implies
  $$\frac{\partial u}{\partial x}=\frac{\partial v}{\partial y}$$
  and $$\frac{\partial v}{\partial x}=-\frac{\partial u}{\partial y}$$
\begin{example}
  Check this works for $P(z)=z^2=(x^2-y^2)+i(2xy)$
  $$DP_{(x,y)}=
  \begin{bmatrix}
    2x & -2y \\
    2y & 2x
  \end{bmatrix}$$
\end{example}

\subsection*{Outline of proof}
\begin{remark}
  $P(z)=\sum_{j=0}^na_jz^j$ of degree $n$, so $a_n\neq 0$, there are at most $n$
  roots of $P$. If $w_0$ a root of $P$, can factor $P$.
    $$P(z)=P_1(z)(z-w_0)$$
    $$deg P_n<n \text{ use I.H.}$$
\end{remark}
\begin{enumerate}[(1)]
  \item  We define from $P(z)=\sum_{j=0}^na_jz^j$ a smooth map
    $$F:S^2\to S^2$$
    Diagram 1
    \newline
    if $a\in S^2$, $a\neq ((0,0,1)=N)$ then $a=\varphi _N(x,y)= \varphi _N(x+iy)$
    \newline define $F(a)=\varphi _N(P(x+iy))$.
    \newline for $N=(0,0,1)\in S^2$, define
      $$F(N)=\begin{cases}
        N \text{ if } P \text{ is not constant} \\
        a_0 \text{ if } P(z)=a_0
      \end{cases}$$
\end{enumerate}
\begin{enumerate}[(1)]
  \item Construct $F:S^2 \to S^2$
  \item At any regular value
    $$y\in S^2, \left(\forall x\in f^{-1}(y), DF_x \text{ is surjective.}\right)$$
    we know $F ^{-1}(y)\subseteq S^2$ is a manifold of dimension $0$.
    \newline A $0$-manifold is a \emph{discrete subset of $S^2$}. i.e. $\{x\}\subseteq f^{-1}(y)$ is open for all $x\in f ^{-1}(y)$.
    \newline $f^{-1}(y)$ is closed set, $\{y\}\subseteq S^2$ is closed, $f ^{-1}(y)\subseteq S^2$ is thus compact. So $f ^{-1}(y)$ is a finite set.
      $$\{x\}|x\in f^{-1}(y)\}$$
    open cover imp finite set.
\end{enumerate}
\begin{enumerate}[(1)]
  \item  Construct $F:S^2 \to S^2$
  \item  At any regular value $y\in S^2$, har $F^{-1}(y)=\{x_1,\dots ,x_N\}$ a finite set.
  \item What are critical points of $F?$
  i.e. $DF_P$ not surjective? Possibly $\infty$, otherwise, it is the points where
    $$P'(z)=0 \leftrightarrow DF_{\varphi _N(z)}=0$$
  at most $n-1$ points
    $$z_1,\dots,z_{n-1}$$
  So regular values of $F$ are at least
    $$S^2\setminus\{F(\infty),F(\varphi _N(z_1)),\dots,F(\varphi _N(z_{n-1})\}$$

\end{enumerate}
\begin{enumerate}[(1)]
  \item  Construct $F:S^2 \to S^2$
  \item  At any regular value $y\in S^2$, har $F^{-1}(y)=\{x_1,\dots ,x_N\}$ a finite set.
  \item What are critical points of $F?$
    $$S^2\setminus\{F(\infty),F(\varphi _N(z_1)),\dots,F(\varphi _N(z_{n-1})\}$$
  \item Define a function
    $$C:S^2\setminus\{F(\infty),F(\varphi _N(z_1)),\dots,F(\varphi _N(z_{n-1})\}\to \N$$
    $$C(y)=\#F^{-1}(y)$$
  \item Technical Theorem (Stack of Records Thm.)
  \newline Says $C$ is continuous/locally constant.
  \item $$S^2\setminus\{F(\infty),F(\varphi _N(z_1)),\dots,F(\varphi _N(z_{n-1})\}$$
  is connected so $C$ is globally constant
    $$\forall y, C(y)=c \in \N$$
  \item
    \begin{enumerate}[Case (1)]
      \item $C(y)=0$
      \newline For almost all $y\in S^2$, $F^{-1}(y)=\emptyset$.
      So image of $S^2$ is not a regular value.
        $$F(S^2)\subseteq \{\infty,F(\varphi _N(z_1))\dots F(\varphi _N(z_{n-1}))\}$$
      So $F(S^2)=\{\varphi _N(a_0)\}$ is a constant function.
      \item Say $C(y)>0$.
      \newline Note that $\varphi _N(0)=(0,0,-1)=S$.
      \newline Either $S\in \{F(\varphi _N(z_1))\dots F(\varphi _N(z_{n-1}))\}$ in which case $S=F(\varphi _N(z_i)) \Rightarrow 0=P(z_i)$.
      \item $C(y)>0$ but now $S$ is a regular value. So $F^{-1}(S)$ has $C(y)>0$ elements. So there is some point $p\in S^2$, with $F(p)=S, p\neq N$ so
        $$P(\varphi ^{-1}_N(p))=0$$
        \qedhere
    \end{enumerate}
\end{enumerate}

\begin{lemma}
  If $X$ is discrete and compact, $X$ is finite.
\end{lemma}
\begin{proof}
  Cover $X$ with
  %curly u
    $$u=\{\{x\} |x\in X \}$$
  only subcover is $u$ itself.
\end{proof}

\begin{theorem}
  \textbf{Stack of records theorem}
  \newline $f: X \to Y$ smooth,
  \newline $dim X = dim Y$
  \newline $f$ is a submersion at all points. $X$ is compact. Then for any $y\in Y$, $f^{-1}(y)$ is a finite set, $f^{-1}(y)=\{x_1,\dots,x_N\}$ and there exists open sets
    $$x_i \in u_i \subseteq X, 1\leq i \leq N$$
    $$y\in V \subseteq Y$$
    Drawing
  \begin{enumerate}[(1)]
    \item $f\rvert_{u_i}: u_i\to V$ is a diffeo
    \item $u_i \cap u_j=\emptyset$ if $i\neq j$
    \item $f^{-1}(V)=\cup_{i=1}^N u_i$
  \end{enumerate}
    $$f^{-1}(y)={x_1,\dots, x_N}$$
    $$\forall z\in V, f^{-1}(z)=\{f^{-1}\rvert_{u_1}(z),f^{-1}\rvert_{u_2}(z),\dots,f^{-1}\rvert_{u_N}(z)\}$$
\end{theorem}
\begin{proof}
  $f^{-1}(y)$ is a $0$-manifold, as $y$ is a regular valure for $f$, $dim X = dim Y$.
  \newline Compactness of $X \Rightarrow f^{-1}(y)$ is compact, so it's finite, $f^{-1}(y)=\{x_1,\dots,x_N\}$.
  \newline First off, $Df_{x_i}:T_{x_i}X \to T_yY$ is surjective $\Rightarrow Df_{x_i}$ is isomorphism
    $$\Rightarrow f \text{ is a local diffeo at }x_i$$
  So we find $x_i\in u_i\subseteq X$ open, $y\in v_i\subseteq Y$ open
    $$f\rvert_{u_i}:u_i \to v_i \text{ diffeo.}$$
  Take $V=\cap_{i=1}^N v_i$. Redefine $u_i=f\rvert_{u_i}^{-1}(V)$.
  \newline How to ensure $u_i \cap u_j=\emptyset$ when $i\neq j$?
  \newline There are open balls $B^n(x_i;\varepsilon)$, so that
    $$B^n(x_i;\varepsilon)\cap B^n(x_j;\varepsilon)=\emptyset \text{ when }i\neq j$$
    $X$ is compact, $X\setminus \left(\cup_{i=1}^Nu_i\right)$ closed, so compact.
      $$f(X\setminus \left(\cup_{i=1}^Nu_i\right)) \text{ is compact},$$
      $$Y\setminus f(X\setminus \left(\cup_{i=1}^Nu_i\right))\text{ open}.$$
    Redefine $V=V\cap \left(Y\setminus f(X\setminus \left(\cup_{i=1}^Nu_i\right))\right)$
    \newline Redefine $u_i=f^{-1}\rvert_{u_i}(V)$.
\end{proof}
